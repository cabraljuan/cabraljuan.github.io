%%%%%%%%%%%%%%%%%%%%%%%%%%%%%%%%%%%%%%%%%
% Medium Length Professional CV
% LaTeX Template
% Version 2.0 (8/5/13)
%
% This template has been downloaded from:
% http://www.LaTeXTemplates.com
%
% Original author:
% Rishi Shah 
%
% Important note:
% This template requires the resume.cls file to be in the same directory as the
% .tex file. The resume.cls file provides the resume style used for structuring the
% document.
%
%%%%%%%%%%%%%%%%%%%%%%%%%%%%%%%%%%%%%%%%%
%comillas “”

%----------------------------------------------------------------------------------------
%	PACKAGES AND OTHER DOCUMENT CONFIGURATIONS
%----------------------------------------------------------------------------------------

\documentclass{resume} % Use the custom resume.cls style
\usepackage[left=0.75in,top=0.6in,right=0.75in,bottom=0.6in]{geometry} % Document margins
\newcommand{\tab}[1]{\hspace{.2667\textwidth}\rlap{#1}}
\newcommand{\itab}[1]{\hspace{0em}\rlap{#1}}
\name{Cabral Juan Andrés} % Your name
%\address{156 Kasturi, Balajinagar, Sangli 416416} % Your address
%\address{123 Pleasant Lane \\ City, State 12345} % Your secondary addess (optional)
\address{Contact: juan.drs.c@gmail.com} % Your phone number and email

\address{Github: \href{https://github.com/jnccabral}{https://github.com/jnccabral}
}
\address{Birth year: 1997} % Your secondary addess (optional)

\usepackage{hyperref}
\hypersetup{
     colorlinks = true,
     linkcolor = blue,
     anchorcolor = blue,
     citecolor = blue,
     filecolor = blue,
     urlcolor = blue
     }

\begin{document}

%----------------------------------------------------------------------------------------
%	EDUCATION SECTION
%----------------------------------------------------------------------------------------

\begin{rSection}{Education}

{\bf Universidad de Buenos Aires} \hfill {\em 2015 - 2020} 
\\ Licentiate in Economics - 8.03/10 - Cum Laude
\\
{\bf Universidad de San Andrés} \hfill {\em 2020- 2021} 
\\ M.A. in Economics 

%Minor in Linguistics \smallskip \\
%Member of Eta Kappa Nu \\
%Member of Upsilon Pi Epsilon \\



\end{rSection}
\begin{rSection}{Publications}
Cabral J. (2019) “La crítica de Bunge hacia la metodología de Friedman 1953” in Scientia in verba Magazine. Volumen 5. Homenaje a Mario Bunge. Edición especial. ISSN: 2619-2586
\end{rSection}

\begin{rSection}{Presentations at conferences}
\begin{itemize}
  \item “¿Cuál es el alcance de la revolución de la credibilidad?” in Asociación Argentina de Economía Política 2020.
  \item “¿Cuál es el alcance de la revolución de la credibilidad?” in Jornadas de Epistemología de las Ciencias Económicas 2020.
 \item “Sobre la metodología de la Escuela Austríaca” in Jornadas de Epistemología de las Ciencias Económicas 2020.
  \item “Bunge y los austríacos” in Forum filosófico internacional “Mario Bunge: Ciencia y Filosofía” Círculo de estudios de filosofía analítica 2020.
  \item “La epistemología detrás de la Economía conductual” in Jornadas de Epistemología de las Ciencias Económicas 2019.
  \item “Una contradicción bajo competencia perfecta” in Jornadas de Epistemología de las Ciencias Económicas 2019.
  \item “¿Qué sabemos hasta ahora sobre la brecha salarial de género?” in Jornadas de Economía Crítica 2018.
  \item “El giro F y dos casos de estudio” in Jornadas de Epistemología de las Ciencias Económicas 2018.

\end{itemize}

\end{rSection}
\begin{rSection}{Research scholarships}
\begin{itemize}
\item EVC-CIN 2020 - Present. Research assistant for María Fernandez.
\item PROPAI 2018 - 2020. Research assistant for Daniel Aromí.
\end{itemize}
\end{rSection}



\begin{rSection}{Teaching experience}
\begin{itemize}


\item {\bf Facultad de Ciencias Económicas - Universidad de Buenos Aires}
\\ Undergraduate
\\Mathematical Analysis 2 \hfill {\em 2018 - Present} 
\\Teacher assistant.
\\Dictation of classes

\item {\bf Facultad de Ciencias Económicas - Universidad de Buenos Aires} 
\\ Underaduated
\\Economics Epistemology \hfill {\em 2018 - Present} 
\\Teacher assistant. 
\\Dictation of classes and correction of practical work.
\item {\bf Universidad Nacional de Quilmes} 
\\ M.A. in Philosophy
\\Philosophy of Economics \hfill {\em 2019} 
\\Teacher assistant
\item {\bf Facultad de Ciencias Económicas - Universidad de Buenos Aires} 
\\ Undergraduate
\\Microeconomics 1 \hfill {\em 2017 - 2020} 
\\Teacher assistant.
\\Dictation of classes and correction of practical works and exams
\end{itemize}

\end{rSection}

\begin{rSection}{Courses taken}
\begin{itemize}
\item “Foundations of Development Policy: Advanced Development Economics” MIT  {\bf Final grade: 81/100}. 2020
\item “Political Economy and Economic Development” MIT {\bf Final grade: 81/100}. 2020
\item “Econometría”, dictated by Walter Sosa Escudero. Final work in R. {\bf Final grade: 9/10}. 2019
\item Workshop “Técnicas cuantitativas en R” dictated by Martin Masci and Rodrigo del Rosso. 2019 
\item “Curso intensivo de Deuda Soberana” dictated by Martín Guzman. 2019
\item “Introduction to Python” Datacamp. 2019
\item “Intro to statistics with R: introduction” Datacamp. 2019
\item “Introducción al análisis de datos utilizando Python” Math department. Final work in Python. 2019
\item “Scientific philosophy” Relativistic Astrophysics Universidad Nacional de La Plata dictated by PhD Gustavo Romero. {\bf Final grade: 9/10}. 2018
\item Cornell Alliance for Science Latin America Leadership Course on Strategic Planning and Effective Grassroots Organizing, Cornell University. CIMMYT, El Batán, México. 2016
\end{itemize}
\end{rSection}



%--------------------------------------------------------------------------------
%  Becas
%-----------------------------------------------------------------------------------------------





\begin{rSection}{Non-academic Articles}
\begin{itemize}
\item {\bf Articles}
\\ “Compassionate lifestyles are backed by scientific facts.” in \url{http://www.vegangmo.com/} \hfill {\em 2017}
\\ “GMOs” for the Animals: The case of pig-free insulin” in \url{http://www.vegangmo.com/} \hfill {\em 2017}
\\ “Somos menos capaces de lo que creemos” in \url{https://circuloesceptico.com.ar/} \hfill {\em 2017}  
\\ “¿Por qué dejar de comer animales? La sintiencia tenida en cuenta” \\ in “Locura Vegana Feria: El libro” \hfill {\em 2017} 
\\ “Holofonía y la pseudociencia en el audio” in \url{https://circuloesceptico.com.ar/} \hfill {\em 2016} 
\\ “Cómo buscar información fiable” in \url{https://circuloesceptico.com.ar/} \hfill {\em 2015} 
\item {\bf Translations}
\\ “Manifiesto Vegano: ¿Por qué debemos apoyar los alimentos transgénicos y la investigación en biotecnología?” in \url{http://www.siquierotransgenicos.cl/} \hfill {\em 2016} 
\\ “¿En dónde se siembran y dónde están prohibidos los cultivos transgénicos?” in \\ \url{http://www.siquierotransgenicos.cl/} \hfill {\em 2016} 



\end{itemize}

\end{rSection}

\begin{rSection}{Coding}
\begin{itemize}
\item  (in process) \href{https://drive.google.com/file/d/1yt3uI4X8O5Wjk_SEH01iFxirgr1q1j6q/view?usp=sharing}{R Applied Economics cookbook}
\item \href{https://github.com/jnccabral/Using-words-as-regressors}{Python sample}
\item \href{https://github.com/jnccabral/Di-tella-and-Schargrodsky-2004-replication}{Stata sample}
\end{itemize}
\end{rSection}




\begin{rSection}{Languages}
{\bf Spanish} - Native\\
{\bf English} - First Certificate in English
\end{rSection}

%--------------------------------------------------------------------------------
%   OTROS
%-----------------------------------------------------------------------------------------------


\begin{rSection}{References}
I worked with María Fernandez as a research assistant, Pablo Fajfar supervised my undergraduate thesis and I worked with Diego Weisman in two courses.
\begin{itemize}
\item María Fernandez - Professor of Mathematical Analysis 2 at Universidad de Buenos Aires \\
 \href{mailto:mariaj.fernan$@$gmail.com}{mariaj.fernan$@$gmail.com} %- +54 9 11 6122-0828
\item Pablo Fajfar - Professor of Behavioral Economics and Mathematics for economists at Universidad de Buenos Aires \\
\href{mailto:pffajfar$@$yahoo.com.ar}{pffajfar$@$yahoo.com.ar} %- +54 9 11 6159-6172
\item Diego Weisman  - Professor of Epistemology of Economics at Universidad de Buenos Aires \\ \href{mailto:diego_mw$@$hotmail.com}{diego\textunderscore mw$@$hotmail.com} %- +54 9 11 3674-4922
\end{itemize}
\end{rSection}


\begin{rSection}{Others}


-Participation in III University Olympics in Economics, organized by Consejo profesional de Ciencias Económicas de la ciudad Autónoma de Buenos Aires, Level 3. 2019.
\\
-Compiler in “Jornadas XXV de Epistemología de las Ciencias Económicas”.
\\
-Part of the organization in “Taller interdisciplinario en sistemas complejos” Instituto Interdisciplinario de Economía Política de Buenos Aires (IIEP) FCE-UBA.
\\
-Asistence to “Primer Encuentro Latinoamericano de Filosofía Científica” 2015.
\\
-Asistence to conference of “Redacción académica” dictated by PhD Pablo Herrera
\\
-Participated in seminar “Expectativas Racionales, credibilidad e inflación” 
\\
-Assistance to conference “Cómo hacer la tesis para el seminario de integración y aplicación” dictated by PhD Pablo Herrera 
\\
-Assistance to conference “Taller de investigación: Preguntas, objetivos e hipótesis” dictated by Pablo Herrera
\\
-Assistance to a Expopyme 2017 as representator of Soylent ARG.
\\
-Assistance to “Jornadas FINCA de soberanía alimentaria” 2016.
\\
-Assistance to conference “Metodología cuantitativa en ciencias económicas” dictated by PhD Pablo Herrera
\\
-Assistance to conference “Distribución de la riqueza: la desigualdad a nivel mundial” dictated by Phd Branko Milanovic 
\\
-Assistance to conference “Introducción a las APIs y programación” dictated by Sebastian Buffo Sempe 
\\
-Part of the organization in “Escepticismo científico” Buenos Aires. 2019
\end{rSection}








\end{document}
%----------------------------------------------------------------------------------------
%	TECHNICAL STRENGTHS SECTION
%----------------------------------------------------------------------------------------

\begin{rSection}{Technical Strengths}

\begin{tabular}{ @{} >{\bfseries}l @{\hspace{6ex}} l }
Modeling and Analysis \ & AutoCad, Revit, StaadPro \\
Software \& Tools & MS Office, Latex \\
\end{tabular}

\end{rSection}

%----------------------------------------------------------------------------------------
%	WORK EXPERIENCE SECTION
%----------------------------------------------------------------------------------------

\begin{rSection}{Work Experience}

\begin{rSubsection}{SJ Contracts, Pune}{June 2016}{Site Engineer}{}
\item On-site internship under this leading construction company. Learned and implemented various aspects such as quantity estimation, labour management and safety precautions.
\end{rSubsection}


\end{rSection}


%	EXAMPLE SECTION
%----------------------------------------------------------------------------------------

\begin{rSection}{Academic Achievements} 
 Runners up in B.G.Shirke Vidyarthi Competition for Innovative Project organized by Pune Construction Engineering Research Foundation in January 2018
\item Won First Prize in Model Making Competition Organized by Symbiosis Institute of Technology, Pune.
\end{rSection}

\newpage

%----------------------------------------------------------------------------------------
% Extra Curricular
%----------------------------------------------------------------------------------------
\begin{rSection}{Extra-Cirrucular} \itemsep -3pt
\item Co-Organized “ Nirmitee 2017” - a National Symposium of Civil Department of MIT, Pune
\item Attended a workshop on Autodesk Revit at IIT Bombay in 2014.
\item Winner of Inter Departmental Football Competition 2015.
\item Member of the  Rotaract Club Of Pune Pride from 2014 to 2017.
\item Worked for a start-up company Named OUST as a Regional Marketing Manager
%\item Trained and disciplined in National Cadet Corps (NCC), IIT Kanpur for a year.
 %\item  Participated in Vijyoshi Camp 2012 organized at Indian Institute of Science, Bangalore.
 %\item Won 2nd position in Kho-Kho in Intramurals conducted by Physical Education Section, IIT Kanpur.
 %\item Pursued French as second language during secondary school from Grade 6 to Grade 10. Also participated in French Song Competition and French G.K. Quiz in Class 10th. %

\end{rSection}

\begin{rSection}{Personal Traits}
\item Highly motivated and eager to learn new things.
\item Strong motivational and leadership skills.
\item Ability to work as an individual as well as in group.
\end{rSection}
